\iffalse

\chapter{Schaltfelder im Stromnetz}
\label{cha:Schaltfelder}

\section{Inspektion von Schaltfeldern im Stromnetz der TWS Netz GmbH}

Ein wichtiger Tätigkeitsbereich in der TWS Netz GmbH betrifft die Schaltfelder. Diese haben die Aufgabe, dass sie kleinere Gebiete im Stromnetz versorgen. 
Dabei hat jedes Schaltfeld einen eigenen Transformator, der für die Stromversorgung sorgt. Eine der Hauptaufgaben ist es, diese Schaltfelder zu inspizieren, 
um zu gewährleisten, dass jedes intakt ist und mit dem richtigen Transformator verschalten ist. Sollte dies nicht der Fall sein, muss dass Schaltfeld geändert 
werden. Änderungen können aber auch baubedingt vorgenommen werden und müssen ebenfalls überwacht und nach den Baumaßnahmen zurückgeschalten werden. Es kann 
allerdings auch eine neue Verschaltung vorgenommen werden, wenn es von Nöten ist. Diese Aufgabe der richtigen Aufteilung von einem in zwei Schaltfelder 
gehört auch zum Aufgabenbereich des Stromnetzmonteurs und wird oftmals in der Niederspannung durchgeführt. Zu diesen Aufgaben gehören \zB die Auftrennung, 
die Dokumentierung und die Erneuerung der Informationsmedien. Schaltfeldänderungen im Mittelspannungsnetz werden nur durchgeführt, wenn ein Stück der 
Leitung durch Baumaßnahmen ausgeschalten werden muss. Die Aufgabe besteht dann darin die Schaltung durchzuführen und ggf. noch ein Mittelspannungskabel zu 
schneiden. Grundlegend soll erlernt werden, wie sich die Schaltfelder untereinander Verhalten und welche Dinge wichtig zu beachten sind, um bei jeder 
Aufgabenstellung eine möglichst geringe Netzbeeinträchtigung zu haben. 

\section{Effiziente Schaltfeldverwaltung im Stromnetz der TWS Netz GmbH}

Das Niederspannungsnetz der TWS Netz GmbH unterteilt sich in eine Vielzahl von Schaltfeldern. Jedes Schaltfeld hat die Aufgabe einen kleinen Bereich im Gebiet 
abzudecken. Dabei teilt sich das Netz in viele verschiedene Zweige, mit Hilfe von KVS und Umspannstationen auf. Da jeder KVS nur eine begrenzte Anzahl an 
Leisten hat, wie in Kapitel 2 erläutert und jeder Transformator nur eine begrenzte Leistung liefert, muss das Stromnetz unterteilt werden. Diese Unterteilung wird 
realisiert durch Schaltfelder, welche voneinander getrennt sind. Die Trennung der Schaltfelder untereinander spielt eine wichtige Rolle, da es im Falle 
einer Störung nur ein bestimmtes Schaltfeld betrifft. Somit kann die Störquelle eingegrenzt werden und es sind nur wenige Haushalte betroffen. Diese 
Auftrennung wird meist in KVS vorgenommen, da diese an andere Schaltfelder angrenzen. In diesen KVS finden sich nicht nur die Kabel des zugehörigen 
Schaltfeldes, sondern meist auch noch ein Kabel des angrenzenden Schaltfeldes. Dies hat den Hintergrund, dass bei einem Störungsfall oder Bauarbeiten die 
Leistung des Schaltfeldes erhöht oder gar ersetzt werden kann, wenn \zB arbeiten am Transformator stattfinden und dieser sein Schaltfeld nicht versorgen kann. 
Jeder Transformator hat eine vorgegebene Nennleistung, mit der er bemessen wird. Diese Nennleistung bringt zur Aussage, wie viel Leistung ein Transformator übertragen kann, 
ohne etwas zu verbrauchen. Diese Leistung wird als Scheinleistung bezeichnet und hat die Einheit kVA. Diese beinhaltet die Wirkleistung und die Blindleistung. 
Die Wirkleistung bezieht sich auf die Leistung, die ein Verbraucher im Betrieb benötigt und wird in Watt (W) angegeben. Diese Leistung bezieht sich auf 
einen reellen Verbrauch, heißt auf einen rein ohmschen Widerstand. Die Blindleistung hingegen ist die verbrauchte Leistung durch Verschiebung der Phase und 
wird in var angegeben. %cite 5 
Diese entsteht an der Ausgangsseite des Transformators durch unsymmetrische Verbraucher und kann variieren. Um einen Transformator richtig zu
bemessen in der Leistung wird dieser in VA angegeben, um die Verluste durch Blindleistung mit einzubeziehen. Je nach Größe des Transformators hat dieser 
unterschiedlich hohe Bemessungen, welche in der Dimensionierung des Schaltfeldes beachtet werden müssen. Die Dimensionierung beschreibt die Festlegung 
bestimmter Größen eines technischen Produkts, um die geforderten Probleme zu erfüllen. Diese entscheidet letztendlich darüber, wie viele Verbraucher über 
das Stromnetz an den Transformator angeschlossen werden können, ohne diesen zu überlasten. Ein Transformator kann grundsätzlich in einer Notsituation überlastet werden, 
allerdings funktioniert dies nur über einen kurzen Zeitraum und in einem gewissen Maß. Eine zu lange Überlastung würde zu einer Überschreitung der 
Grenztemperatur führen und hätte zur Folge, dass die Isolierfähigkeit des isolierenden Öls im inneren des Transformators abnimmt. %cite 10 
Daher ist es im alltäglichen
Betrieb des Stromnetzes von hoher Relevanz die Umspannstationen so zu managen, dass jeder Transformator unter seiner Nennleistung arbeitet. Bei einem zu groß 
werdenden Schaltfeld, durch ein \zB dazu kommendes Neubaugebiet, muss die vor Ort betroffene Umspannstation auf die maximale Nennleistung überprüft werden 
und entweder vergrößert, erweitert oder ergänzt werden durch eine neue Umspannstation. Eine Erweiterung dieser Art zieht eine Änderung des Schaltfeldes
mit sich, um die Verbraucherleistung neu aufzuteilen. Diese Änderung muss zuerst im Schaltfeldplan angepasst werden, um jedem Mitarbeitenden des 
Stromnetzes eine aktuelle Auskunft zu bieten. In diesem Schaltfeldplan sind alle Schaltfelder des gesamten Stromnetzes eingezeichnet und farblich 
voneinander getrennt, um die einzelnen Stromkreise zu unterscheiden. Somit muss ein neues Schaltfeld eine neue farbliche Kennung bekommen und muss an 
den angrenzenden Stromkreisen aufgetrennt werden. Dieses sogenannte Auftrennen beschreibt den Prozess, in dem die Sicherungen im KVS entfernt werden, 
an der Stelle wo das Kabel des neuen Schaltfeldes angeschlossen ist. Somit wird die Verbindung zur Sammelschiene unterbrochen und das Kabel wird nur 
noch von der neuen Ust. aus versorgt. Anschließend müssen an den geänderten KVS und Ust. die Stationskarten/KVS-Karten ausgetauscht werden, um auch dort
ersichtlich zu machen, wie das neue Schaltfeld aufgebaut ist. Diese Information ist entscheidend, um bei Störungssituationen im Schaltfeld schnell 
einzugrenzen, wo sich der Fehler befinden könnte. Eine sogenannte Schaltzustandsstörung wäre ein Beispiel für eine solche Art der Störung. Hierbei wurde
eine ausgelöste Sicherung festgestellt oder eine Störung der Stromversorgung vom Kunden gemeldet, in der allerdings unbekannt ist, wo sich der Fehler 
befindet. Um eine solche Art der Störung zu beheben ist es wichtig ein ersichtliches Schaltfeld vorzufinden, um die Ursache auf ein bestimmtes Schaltfeld 
zu reduzieren. Zudem hilft die Selektivität des Schaltfeldes den Fehler weiter einzudämmen, da in einem selektiven Netz die Sicherungen zum Verbraucher
kleiner dimensioniert sind. Dies bedeutet, dass für ein Kabel, welches von einer Umspannstation weggeht zu einem KVS eine höhere Sicherung, heißt 
gegen einen höheren Strom \zB 250 A abgesichert ist und ein Kabel, welches zu einem Hausanschluss geht nur mit \zB 160 A eingesichert ist. Dieser 
Unterschied beschreibt die Selektivität und sorgt dafür, dass im Stromnetz möglichst nahe an der Fehlerstelle die Sicherung auslöst und somit umliegende
Verbraucher geschützt sind. Zudem wird das Netz rund um eine Ust. im Maschennetz betrieben, was den Vorteil bringt, dass bei einem Kabelausfall 
durch Beschädigung oder Erneuerung der jeweilige KVS von mehreren Verbindungsseiten aus versorgt wird und jedes Kabel ebenfalls von beiden Seiten
an die Versorgung der Ust. oder des nächsten KVS angeschlossen ist. Somit ist immer eine Versorgung von mehreren Anschlussseiten gewährleistet, 
welches die Störanfälligkeit zusätzlich senkt. Ein Maschennetz beschreibt eine Stromnetzart, in der jeder Knotenpunkt, hier als KVS oder Ust. bekannt, 
von mehreren Kabeln versorgt wird. Somit kann bei einem Kabelausfall jeder KVS weiterversorgt werden, da er von anderen Stromkreisen, den Maschen 
ebenfalls versorgt wird. Vereinzelt verwendet man das Prinzip des Strahlennetzes, welches nur von einem Knoten aus versorgt wird und wie ein Strahl
verläuft. Kommt es bei einem solchen Netztyp zu einer Störung, fällt meist der Strom auf der gesamten Länge des Strahls aus, da diese Masche an 
keinem zweiten Knoten angebunden ist und somit auch nicht von dort versorgt werden kann. %cite 11
\\\\
Bei den Schaltkreisen im Mittelspannungsnetz ist das Management ein wenig komplizierter, da diese für die Hauptversorgung der Niederspannungsschaltfelder
verantwortlich sind und eine der wichtigsten Knotenpunkte im Stromnetz darstellen. Aus diesem Grund ist es essentiell wichtig, diese im Ringnetz oder
Maschennetz zu erbauen, um auch bei Störungen oder Bauarbeiten eine Versorgungssicherheit zu gewährleisten. Bei einem Ringnetz hängt jede Masche auf zwei
Knoten und kann somit ähnlich wie beim Maschennetz in abgewandelter Form von zwei Knoten aus versorgt werden. Änderungen in diesem Netz, \zB durch 
Baustellen müssen immer so geplant werden, dass nur ein Teilstück des Ringes ausgeschaltet wird, um eine Versorgung weiter zu garantieren. Zudem sollte
man in den Schaltplänen überprüfen, ob es an einer Ust. oder einem Schaltwerk in der Nähe der Baustelle eine offene Verbindung zu einer anderen
Masche gibt, um diese ggf. dazuzuschalten. Dadurch kommt eine zusätzliche Verbindung im Knoten hinzu, welche bei unerwarteten Störungen größere 
Stromausfälle vermeiden kann, da der Knoten von mehreren Maschen versorgt wird. %cite 11 [Bild MS Netz]
\\\\
Grundsätzlich muss eine Schaltung im Mittelspannungsnetz beim Netzbetreiber beantragt werden. In diesem Fall betrifft dies den Antrag bei der Netze BW GmbH.
Diese überprüfen anschließend ob eine Schaltung an den gewünschten Stellen möglich ist und geben diese dann frei. Dieser sogenannte Schaltantrag beinhaltet
nicht nur die Genehmigung der Schaltung, sondern auch einen genauen Ablauf, wie diese stattzufinden hat. Darunter zählen die Tätigkeiten, des ein- oder 
ausschalten des Lasttrennschalters, wie auch das einlegen oder entfernen der Erde. Die Schaltung des Lasttrennschalters bringt mit sich, ob das 
angeschlossene Mittelspannungskabel mit der Sammelschiene verbunden ist und unter Spannung steht oder nicht. Bei einem ausgeschalteten Kabel, muss der
Schalter für die Erde eingelegt werden. Dieser funktioniert gleich wie ein Lasttrennschalter, nur das dieser dafür sorgt, dass das Kabel mit der Erde
verbunden ist und somit sämtliche Fehlerströme in das Erdreich abgeleitet werden. Da dieses Netz im Ring- oder Maschennetz betrieben wird, reicht es 
nicht aus, den Lasttrennschalter von einer Seite auszuschalten. Dieser muss immer auf beiden Enden des Kabels ausgeschaltet werden, um eine 
Spannungsfreiheit herzustellen. Diese muss anschließend mit einem Mittelspannungsprüfer überprüft werden, um sicherzustellen, dass diese auch wirklich 
vorliegt. Ein Mittelspannungsprüfer ist ein Messgerät zur Feststellung der Spannung an Anlagen bis zu einer Nennspannung von 36 kV. Die Messspitze wird 
hierfür an die Phasen der Sammelschiene gehalten und zeigt anschließend an, ob eine Spannung anliegt. %cite[12][Bild_MS_Prüfer] 
Nach feststellen der 
Spannungsfreiheit können nun arbeiten am Kabel oder in der Nähe des Kabels durchgeführt werden. Diese beziehen sich oftmals auf die Verlegung neuer
Kabel, arbeiten von anderen Bauunternehmen in der Nähe des Kabels oder das Schneiden eines alten oder auszutauschenden Kabels.\\\\
Das Schneiden eines Mittelspannungskabels funktioniert mit Hilfe einer Sicherheitsschneidanlage. Diese Anlage besteht aus einem hydraulischen Schneidkopf
und einer zugehörigen Pumpe. Dies bedeutet, dass die gesamte Anlage mit einem nichtleitenden Öl betrieben wird, welches durch erhöhen des Drucks dafür 
sorgt, dass sich der Schneidkopf schließt. Durch Schließen des Schneidkopfes wird das zu schneidende Kabel durchtrennt. %[Bild_MS_Schneidanlage]
Im Falle 
eines unter Spannung stehenden Kabels, sorgt das nichtleitende Öl dafür, dass keine Gefahr für Mitarbeitende entsteht und diese trotzdem das Kabel 
durchtrennen können. Vorausgesetzt wird, dass dieser Fall nicht eintritt. Dafür werden Maßnahmen getroffen, die die Überprüfung der Netzpläne beinhalten,
um sicherzustellen, dass dieses Kabel inaktiv ist. Zudem muss vor jedem Kabelschnitt Rücksprache mit der zuständigen Leitstelle gehalten werden, welche 
zusätzlich die Spannungsfreiheit und die Zulassung zur Schneidung überprüft und freigibt. Die Leitstelle ist eine Zentrale, in der das komplette Stromnetz
der Umgebung überwacht wird und über jegliche Störung informiert wird. Sie ist Informationsempfänger und Vermittler für sämtliche Anliegen rund um ihr 
Einsatzgebiet und ist der erste Ansprechpartner für Mittelspannungsanliegen. 

\section {Schaltfeldmanagement für eine zuverlässige Stromversorgung}

Die Schaltfelder im Niederspannungsnetz haben einen großen Einfluss auf die Effektivität und Funktionalität des Stromnetzes. Um diese Eigenschaften jederzeit 
zu gewährleisten, ist es von Nöten diese Schaltfelder zu überwachen und zu managen. Zudem muss jedes Schaltfeld so zusammengestellt werden, dass selbst bei 
einem unerwarteten Ausfall eine Versorgungssicherheit hergestellt ist. Diese Sicherheit kommt vor allem aus den genannten Punkten der maximalen Nennleistung 
eines Transformators und den verschiedenen Netztypen. Die Nennleistung ist hinsichtlich einer Notsituation von enormer Relevanz, da diese entscheidend ist für ein 
stabiles Netz. Wie im Kapitel 4.2 erläutert, kann ein solcher Transformator auch über kürzere Zeit höhere Leistungen erbringen, welche meist nicht von Nöten sind, 
da die Schaltfelder so aufgeteilt wurden, dass genügend freie Leistung zur Verfügung steht. Sollte ein Transformator an seine Grenzen kommen, gibt es die Möglichkeit 
diesen durch einen größeren auszutauschen oder das Schaltfeld aufzuteilen und eine neue Ust. zu bauen. Nicht nur der Transformator sorgt für ein störungsfreies Netz, 
sondern auch die Selektivität. Diese hat die Aufgabe, dass nur kleine Stücke vom Netz herunterfliegen, wenn es zu einer Störung kommt. Zudem sorgt Sie dafür, 
dass man die Störung schnell eingrenzen kann, da durch ein selektives Netz die Sicherung, welche sich am nächsten zur Störung befindet auslöst. Dadurch ist
es möglich Störungen schnell zu finden und zu bearbeiten, um dem Kunden lange Störungszeiten zu verhindern. Auch der Netztyp, welcher im Kapitel 4.2 
vorkommt trägt zu einem stabilen Netz bei. Dies liegt vor allem an der Versorgung von zwei oder mehr Seiten, wie es im Ring- oder Maschennetz der Fall 
ist. Dadurch ist es zusätzlich möglich Baustellen zu realisieren, ohne dem Netzkunden eine Unterbrechung zuzumuten. Das Mittelspannungsnetz ist ein 
Hauptakteur, wenn es um die Versorgung des Niederspannungsnetzes geht und ist Grundvoraussetzung für eine funktionierende Stromversorgung. Dazu hat dies 
auch den Netztyp des Ring- oder Maschennetzes, um eine Versorgungssicherheit zu gewährleisten. Diese Sicherheit kann durch ein Strahlennetz nicht erreicht
werden, da es bei einer Störung zu einem Gesamtausfall auf der Länge des Strahls kommen würde. Zudem kann in einem Ring- oder Maschennetz flexibel 
geschaltet werden, um sich den aktuellen Vorkommnissen anzupassen. Die Rücksprache und Beantragung von Vorkommnissen im Mittelspannungsnetz mit der 
Leitstelle tragen zu einer Sicherheit in der Versorgung, als auch beim Mitarbeitenden vor Ort bei, da dieser über die Situation informiert ist und sich
ggf. schützen kann. Dies erfolgt \zB durch zusätzliche Spannungsprüfungen oder speziellem Werkzeug zum schneiden von Kabeln unter Spannung. Dies dient 
vor allem dazu, vor dem Ernstfall geschützt zu sein, falls eine Fehlinformation durch das System übermittelt wurde.  

\clearpage

\fi
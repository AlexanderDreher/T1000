\chapter{Zusammenfassung}
\label{cha:zusammenfassung}

Die Projektarbeit befasste sich mit den Strukturen und Tätigkeiten des Elektrikers für Betriebstechnik in der TWS Netz GmbH. Dazu wurden verschiedenste 
Tätigkeitsschwerpunkte genannt, welche eine große Rolle im Alltag dieses Berufes spielen. Sie dienen nicht nur als Hilfsstellung für jegliche Probleme
im Stromnetz, sondern besitzen zudem eine fördernde Wirkung für eine theoretische, als auch praktische Annahme und Lösung eines Problems. Dazu wurden 
Fertigkeiten im Bereich der Metallbearbeitung geschult, welche das Verständnis für die richtige Wahl des Werkzeuges und dem richtigen Verfahren förderten. 
Diese Entscheidung spielt eine wichtige Rolle in der Lösung von individuellen Problemstellungen, da Sie am Anfang eines Problems getroffen wurde und 
verantwortlich für eine nachhaltige und zielorientierte Lösung ist. Bei einer nicht idealen Entscheidung, kann es zu zeit-, kosten- oder umwelttechnischen 
Konsequenzen kommen, welche das Image des Unternehmens schädigen und zusätzlich ein Misstrauen beim Kunden verursachen. Um solche Entscheidungen genauer 
zu kalkulieren wurden Kenntnisse zur Berechnung mathematischer Theoreme im Bereich Elektrotechnik gefördert. Diese theoretischen Grundlagen wurden angewandt, 
um die praktischen Vorgehensweisen anhand mathematischer Theoreme zu belegen. Dazu zählen überwiegend die Berechnung von Stromkreisen, Leitungswiderständen 
und Trafoleistungen. Diese Kennwerte tragen zu einem stabilen und zuverlässigen Stromnetz bei und dienen zusätzlich als Grundlage zur Planung neuer 
Umspannstationen oder Schaltfelder. Somit wird nachhaltig für eine Versorgungssicherheit im Netzgebiet der TWS Netz GmbH gesorgt. Es reicht allerdings 
nicht aus, das Stromnetz auf Grundlage von theoretischen Kennwerten zu managen, da die Einflüsse der Umwelt ebenso große Anteile tragen und beseitigt 
werden müssen. Es ist zwar möglich eine ungefähre Lebensdauer \zB eines Holzmasten zu berechnen, allerdings hat die Praxis bewiesen, dass durch äußere 
Einflüsse, diese auch stark abweichen kann. Daher ist es ebenso wichtig das Stromnetz regelmäßig zu kontrollieren, um Abweichungen frühestmöglich zu 
erkennen. Bei Freileitungen können Probleme dieser Art schnell identifiziert und behoben werden, ganz im Gegenteil zu Erdkabeln. Bei diesen müssen 
Strukturen zur ordnungsgemäßen Montage und zur Sauberhaltung der Anlagen eingehalten werden, um dem Kunden ein störungsfreies Netz zu gewährleisten. 
Wichtig sind diese Strukturen vor allem bei Schwachpunkten im Netz, wie \zB den Kabelmuffen. Diese sind besonders anfällig für Netzstörungen, wenn bei 
der Montage nicht auf Genauigkeit geachtet wird. Zudem kann es in diesem Tätigkeitsbereich schnell zur Schädigung der Umwelt kommen, wenn das Erdreich 
durch austretendes Öl von alten Kabeln belastet wird. Des Weiteren ist es wichtig Richtlinien und Sicherheitsvorschriften, vor allem im Bereich der 
Mittelspannung einzuhalten, da es in diesem Bereich schnell zu Personenschäden kommt und die Umwelt stark belastet werden kann. Genannte Beispiele sind 
unter Spannung stehende Kabel oder \ce{SF_6} Schaltanlagen. Ziel ist stets ein stabiles und leistungsfähiges Netz gegenüber dem Kunden zu garantieren. Hierzu 
trägt ein Ring- oder Maschennetz positiv bei und sorgt somit im Nieder-, als auch Mittelspannungsbereich für eine flexible und krisensichere Versorgung. 
All diese Tätigkeiten und Strukturen der TWS Netz GmbH sorgen für eine Verknüpfung der theoretischen Kenntnisse mit praktischen Problemstellungen und 
fördern eine eigenständige Lösung eines Problems unter Anwendung ingenieurstechnischer Vorgehensweisen.
\chapter{Zusammenfassung}
\label{cha:zusammenfassung}

Die Praxisarbeit konzentrierte sich auf die Betriebsaufgaben der Stromversorgung bei der TWS Netz GmbH. Dabei wurden verschiedene Aufgabenbereiche erkundet, 
um die wesentlichen technischen und organisatorischen Prozesse in der Stromversorgung zu erlernen. Diese Prozesse dienen nicht nur als Unterstützung bei 
Stromnetzproblemen, sondern fördern auch die theoretische und praktische Herangehensweise zur Lösung von Problemen. Im Rahmen dieser Arbeit wurden auch 
Fertigkeiten in der Metallbearbeitung entwickelt, was das Verständnis für die richtige Auswahl von Werkzeugen und Verfahren zur Problemlösung förderte. 
Um solche Entscheidungen genauer zu treffen, wurden mathematische Theoreme im Bereich der Elektrotechnik erlernt und angewandt, insbesondere zur Berechnung von 
Stromkreisen, Leitungswiderständen und Transformatorleistungen. Diese Berechnungen tragen zur Stabilität und Zuverlässigkeit des Stromnetzbetriebs bei und 
dienen als Grundlage für die Planung von Erneuerungsbauten, um den Stromnetzbetrieb nachhaltig zu gewährleisten. 
\\\\
Es ist jedoch nicht ausreichend, das Stromnetz auf Grundlage theoretischer Werte zu verwalten, da Umwelteinflüsse ebenfalls eine erhebliche Rolle spielen. 
Regelmäßige Kontrollen des Stromnetzes sind daher entscheidend, um Abweichungen frühzeitig zu erkennen und zu beheben, insbesondere bei Freileitungen. Bei 
Erdkabeln sind präzise Montagestrukturen und die Wahrung der Anlagensauberkeit von besonderer Bedeutung, um eine störungsfreie Stromversorgung sicherzustellen. 
Dies gilt insbesondere für Schwachstellen wie Kabelmuffen, die anfällig für Netzstörungen sind, wenn bei der Montage nicht auf Genauigkeit geachtet wird. 
\\
Die Einhaltung von Richtlinien und Sicherheitsvorschriften, insbesondere im Mittelspannungsbereich, ist von großer Bedeutung, um Personenschäden zu vermeiden 
und Umweltauswirkungen zu minimieren. Das Ziel ist stets, eine stabile und leistungsfähige Stromversorgung für die Kunden sicherzustellen. 
\\\\
All diese Aktivitäten und Strukturen der TWS Netz GmbH fördern die Verknüpfung theoretischer Kenntnisse mit praktischen Herausforderungen und unterstützen 
die eigenständige Lösung von Problemen unter Anwendung ingenieurtechnischer Methoden.

\clearpage

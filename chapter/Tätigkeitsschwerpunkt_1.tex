\chapter{Verfahren zur Metallbearbeitung und Grundlagen der Elektrotechnik}
\label{cha:Metallbearbeitung und Elektrotechnik}

\section{Aufgabenstellung}

Es sind die wichtigsten Grundlagen zur Metallbearbeitung zu erlernen. Dazu sollen zunächst händische Verfahren erlernt werden, um anschließend Methoden zur
maschinellen Bearbeitung kennen zu lernen. Schwerpunkt in dieser Aufgabe besteht darin, Fertigungsverfahren aus dem Bereich Zerspannung, Umformung und Fügung 
an Problemstellungen anzuwenden. Des Weiteren sollen Kenntnisse über die wichtigsten Eigenschaften verschiedener Metallarten erlernt werden. Außerdem ist es 
wichtig, dass Vorschriften zum Arbeitsschutz eingehalten und stets mit bedacht behandelt werden. Ziel der Aufgabenstellung ist es selbstständig eine
Problemstellung unter Zuhilfenahme der erlernten Bearbeitungsverfahren durchzuführen. \\
Anschließend sollen grundlegende Kenntnisse im Bereich der Elektrotechnik erlernt werden. Dazu sollen Problemstellungen zunächst theoretisch erarbeitet 
werden, um diese dann Anhand kleinerer Versuchsaufbaue zu erläutern. Dies findet zunächst im Bereich Gleichstrom statt und soll dann zu Problemen und 
Verfahren im Dreiphasenwechselstrom übergehen. Hier ist es von entscheidender Rolle, dass auch wichtige Regeln und Vorschriften zur Arbeitssicherheit erlernt und 
beachtet werden, um Arbeitsunfälle zu verhindern. Des Weiteren sollen Tätigkeiten und Vorgehensweisen eines Elektrikers geschult werden, um diese an 
Problemstellungen anzuwenden und ein zielorientiertes Arbeiten zu gewährleisten. 

\section{Praktischer Lösungsansatz}

In der Metallverarbeitung gibt es verschiedene Verfahren zur Herstellung eines Werkstückes. Diese Verfahren werden in Hauptgruppen zusammengefasst und 
unterscheiden sich in ihrer Eigenschaft, wie sie Rohmaterialien bearbeiten oder verändern. Eines dieser Verfahren ist das Trennen. Hierbei handelt es sich 
um ein spanendes Fertigungsverfahren.
\begin{description}
\item[Spanende Fertigung] Die spanende Fertigung beschreibt ein Verfahren zur Bearbeitung verschiedener Werkstoffe mit Hilfe von Werkzeugen, bei denen 
Material vom Werkstoff herausgeschnitten wird, um dessen Form oder Oberfläche zu verändern. Das abgetragene Material wird auch als Span bezeichnet. %cite 1
\end{description}
Zu den spanenden Fertigungsverfahren zählen \zB das Feilen, Schleifen, Sägen, Bohren, Drehen und Fräsen. Jedes dieser Verfahren hat seine eigenen 
Eigenschaften und bietet sowohl Vor- als auch Nachteile. 
\\
Das Feilen wird meist von Hand ausgeführt, mit sogenannten Werkstattfeilen und dient zur präzisen Bearbeitung von Werkstücken. Dies hat zur Folge, dass nur 
kleinere Arbeiten mit der Feile getätigt werden können, da andernfalls dieses Verfahren zu zeitaufwendig wäre. Im Gegensatz zur maschinellen Bearbeitung, 
wie \zB beim Fräsen oder Drehen, bietet dass Feilen den großen Vorteil, dass auch filigrane Arbeiten auf engem Raum getätigt werden können. Zudem 
unterscheiden sich Feilen in ihrer Bezahnung, auch Hieb genannt. Es gibt Feilen mit wenigen Hieben, welche ihren Anwendungsbereich in der Bearbeitung 
von weichen Werkstoffen wie Aluminium haben, aber auch zur Grobbearbeitung genutzt werden, um möglichst viel Material abzutragen. Feilen mit einer großen 
Anzahl von Hieben tragen nur wenig Material ab und sind meist ungeeignet für weiche Werkstoffe, da die Späne in den Zwischenräumen stecken bleiben, dafür 
erzeugen diese meist eine glatte Oberfläche mit einer höheren Güte. %cite 2
\\\\
Das Fräsen ist neben dem Drehen eines der wichtigsten Verarbeitungsverfahren zur Bearbeitung von Werkstoffen. Beide Verfahren unterscheiden sich in den 
Anwendungsbereichen und wie sie die Werkstücke bearbeiten. Hierbei sind Werkstücke, die gedreht werden immer symmetrisch, da ausschließlich Runde 
Werkstoffe verarbeitet werden können. Dies liegt daran, dass beim Drehen sich das Werkstück um die eigene Achse dreht und beim Fräsen das Werkzeug. 
Durch diesen Unterschied hat jedes der beiden Fertigungsverfahren seinen eigenen Anwendungsbereich. Das Drehen wird \zB bei Bolzen, Schrauben oder 
Unterlagscheiben angewandt und das Fräsen bei \zB Nuten, Formänderungen oder Bohrungen. Heutzutage unterscheidet man zwischen zwei Arten des Fräsens 
und Drehens, dem konventionellen und dem Computerized Numerical Control (CNC) Fräsen oder Drehen. Beide Verfahren bieten einen sehr hohen Grad an Genauigkeit 
und finden einen großen Anwendungsbereich in der Fertigung präziser Werkstücke. Das Fräsen oder Drehen bringt den großen Vorteil mit sich, dass viel Material 
abgetragen werden kann und die Qualität darunter nicht in Mitleidenschaft gezogen wird. Durch die CNC Technologie ist das Fertigen gleichaussehender Teile 
automatisiert und für den Fließbandbetrieb ideal. Somit bietet dies Unternehmen die Chance Kosten durch schnelle und präzise Fertigung zu reduzieren. 
Allerdings gibt es auch Nachteile beim Fräsen oder Drehen, welche vor allem im thermodynamischen Segment liegen, da die Werkstoffe und Werkzeuge sehr 
großer Hitze ausgesetzt sind und somit die Gefahr herrscht, dass sich die Eigenschaften \zB des Metalls negativ verändern. Um diese thermische Belastung 
einzuschränken, werden oftmals Kühlflüssigkeiten verwendet, welche bei Kontakt eine Belastung für die Umwelt und Gesundheit darstellen. %cite 3 
\\\\
Ein weiteres Verfahren zur Bearbeitung von Werkstoffen, ist die Umformung. Der große Unterschied zu spanenden Fertigungsverfahren ist hierbei, dass kein 
Material entfernt wird. Es wird lediglich die Geometrie verändert, um die gewünschte Form zu erreichen. Da die meisten Metalle die Eigenschaft einer guten 
Verformbarkeit haben, wird dieses Verfahren überwiegend in der Metallindustrie verwendet. Zu solchen Verfahren zählen \zB Walzen, Schmieden und Biegen. Wobei 
in den meisten Unternehmen das Biegen eine größere Rolle spielt, da es einfach in der Anwendung und relativ kostengünstig ist. Beispielsweise könnte mit einem 
Stück Flachstahl ein Winkel erzeugt werden, um etwas zu befestigen. Ein klarer Vorteil kristallisiert sich dabei schnell heraus und zeigt, dass dieses 
Verfahren sehr einfach in der Anwendung und flexibel einsetzbar ist. Allerdings beschränkt sich dies sehr schnell auf einfache Problemstellungen, denn 
sobald ein komplexes Werkstück benötigt wird, reicht dieses Verfahren nicht mehr aus. Ein großer Nachteil ist beim Biegen, dass man einen Mindestbiegeradius 
einhalten sollte, da sich das Material sonst verjüngt oder gar bricht.
\begin{description}
\item[verjüngen] Begriff in der Technik für die Verringerung von Querschnitten im Material
\end{description}
Um dieses Verhalten zu unterbinden, sollte der Biegeradius vor Beginn der Arbeit beachtet werden. Dazu muss je nach Metallart ein Radius von ein oder zweimal 
der Stärke des Metalls genommen werden.\\\\
Das letzte wichtige Verfahren ist das Fügen. Hierbei werden zwei oder mehrere Werkstücke so verbunden, dass sie miteinander eine dauerhafte Verbindung 
erzeugen. Zu den wichtigsten Fügeverfahren zählt das Schweißen, welches in Unternehmen einen großen Anwendungsbereich findet. Sei es in der Verbindung 
und dem Bau von Rohren oder Schiffen, als auch in der Lösung von schnellen Problemen vor Ort, wie \zB zur Reparatur von Beschädigungen. Jeder Einsatzbereich 
hat andere Anforderungen an das Schweißen, was eine Vielfalt an Schweißmethoden und Verfahren voraussetzt. Eines dieser Verfahren ist das 
Lichtbogenhandschweißen, in dem mit Hilfe elektrischen Stroms ein Lichtbogen erzeugt wird, der die Materialien schmilzt und bei anschließender 
Aushärtung miteinander verbindet. Dieses sogenannte Schmelzbad muss durch Zufuhr von einem geeigneten Schutzgas, meist Argon umhüllt sein, um eine Oxidation 
mit dem Umgebungssauerstoff zu verhindern. Diese Oxidation würde zu einer Verschlechterung der Qualität und zu einer spröden Schweißnaht führen, was zur 
Folge hätte, dass diese nicht belastungsfähig wäre. Die Verwendung von Schutzgas wird nur in den Methoden des Metall-Inertgas- (MIG), Metall-Aktivgas- (MAG) 
und Wolfram-Inertgas-Schweißens (WIG) verwendet, da es bei diesen Methoden keine andere Möglichkeit zum Schutz des Schmelzbades gibt. Diese drei Methoden 
bieten den großen Vorteil einer hohen Produktivität, wie auch eine gute Automatisierung, da der Schweißdraht von einer Trommel automatisch und kontinuierlich 
zugeführt wird. Im Gegensatz zu diesen Methoden steht das Elektrohandschweißen mit einer Stabelektrode. Hierbei wird kein Schutzgas benötigt, da sich das 
Schweißbad durch die entstehende Schlacke und den Rauch selber vom Umgebungssauerstoff isoliert. Dies bietet dem Anwender den großen Vorteil, dass diese 
Methode nahezu überall anwendbar ist und keine großen Geräte mit Schutzgaszufuhr benötigen. Deshalb wird diese Methode auch häufig bei Problemstellungen 
im Außenbereich angewandt. Der größte Nachteil ist hierbei die hohe Rauchentwicklung, wie auch der entstehende Aufwand und Dreck bei entfernen der Schlacke. 
Hierzu sollte in geschlossenen Räumen immer eine Absaugung gewährleistet sein, da die Dämpfe gesundheitliche Folgen haben und nicht in großen Mengen 
eingeatmet werden dürfen. Zudem ist es beim Schweißen allgemein von hoher Relevanz, dass ein Augenschutz, wie auch eine geeignete persönliche 
Schutzausrüstung (PSA) getragen wird, um sich vor Funken und Strahlung durch den Lichtbogen zu schützen. \\\\ %\cite 4
Im Folgenden geht es um die Lösung von Problemen im Bereich der Elektrotechnik. Hierzu wird sich der erste Teil auf die Lösung von Gleichstromproblemen und der 
zweite Teil auf die Lösung von Wechsel- bzw. Dreiphasenwechselstromproblemen beziehen. Um einfache Gleichstromkreise zu berechnen, ist es von entscheidender Relevanz, 
die richtigen Formeln anzuwenden. Dazu gibt es \zB Formeln für Parallel oder in Reihe geschaltete Widerstände, die Kirchhoffschen Gesetze oder das ohmsche 
Gesetz. Alle diese Formeln dienen dazu, dass Verhalten von Widerständen zu beschreiben, um daraus praktische Schlüsse in der Anwendung dieser zu ziehen. 
Ein Widerstand hat unter anderem den Nutzen, die Spannung oder den Strom zu verringern, um den Verbraucher zu schützen. Je nachdem, welches Problem zu 
lösen ist, muss der Widerstand parallel, in Reihe oder beides in Kombination verwendet werden. Da es allerdings nur festgelegte Widerstandgrößen zu kaufen 
gibt und meist auch nicht alle im Unternehmen vorhanden sind, müssen verschiedene Größen miteinander kombiniert werden. Durch die Verwendung der Formel für 
parallelgeschaltete Widerstände, kann man \zB durch die Verwendung zweier 100 $\Omega$ Widerstände herausfinden, dass dadurch ein 50 $\Omega$ Widerstand 
entsteht. 
%\clearpage
Dies kann beliebig oft angewandt werden, wobei die Formel 1.1 zur Berechnung von parallelen Widerständen nur für eine maximale Anzahl von zwei 
Widerständen und die Formel 1.2 für eine unbegrenzte Anzahl von Wideständen zählt.
\begin{equation}
R_{\text{ges}}=\frac{R_1 \cdot R_2}{R_1+R_2}
\label{eqn:Parallelschaltung von 2 Widerständen}
\end{equation}
\begin{equation}
\frac{1}{R_{\text{ges}}}=\frac{1}{R_1}+\frac{1}{R_2}+\dots
\label{eqn:Parallelschaltung von mehreren Widerständen}
\end{equation}
Zudem ist bei parallelgeschalteten Widerständen zu beachten, dass die Spannung, welche über den Widerständen abfällt immer gleich bleibt und diese Art der 
Verschaltung nur zu einer Reduktion des Stroms führt. Um den gesamten Strom über den Widerständen zu berechnen, kann folgende Formel angewandt werden.
\begin{equation}
I_{\text{ges}} = \frac{U}{R_{\text{ges}}}
\label{eqn:Gesamtstrom Parallelschaltung}
\end{equation}
Bei einer Reihenschaltung von Widerständen ist die Berechnung deutlich einfacher, da sich diese lediglich addieren. Somit können beliebig viele Widerstände in Reihe geschaltet werden, um den gesamten Widerstand zu erhöhen.
\begin{equation}
R_{\text{ges}}=R_1+R_2+\dots
\label{eqn:Widerstand Reihenschaltung}
\end{equation}
Allerdings ist bei einer Reihenschaltung zu beachten, dass eine Reduktion der Spannung über den Widerständen stattfindet, weshalb dieser Typ Verschaltung 
angewandt wird bei Verbrauchern, die eine geringere Spannung benötigen, als die anliegende. Zudem ist es möglich beide Typen der Verschaltung zu kombinieren. 
Hierbei ist dann jeweils zu beachten, welche der Formeln angewandt werden muss, da beide Typen vorhanden sind. Wichtig dabei zu beachten ist, dass das 
Schaltbild in einzelnen Teilschritten berechnet wird und man die beiden Formeln für die Reihen- und Parallelschaltung nicht vermischt. 
Allgemein gilt immer, dass man von innen nach außen rechnet und versucht am Ende auf einen Widerstand zu kommen, über dem die Spannung oder der Strom abfällt.
%\clearpage
Im folgenden Beispiel wird eine solche Schaltung nochmals genauer erläutert.
\begin{figure}[hbt]
    \centering
    \includegraphics[width=0.8\linewidth]{images/Gemischte Schaltung}
    \caption[Gemischte Schaltung]{Gemischte Schaltung}
    \label{fig:Gemischte Schaltung}
\end{figure}
\\Hier ist es wichtig zuerst die Parallelschaltung zwischen $R_2$ und $R_3$ zu berechnen, um einen Gesamtwiderstand zu bekommen. Mit Hilfe dieses 
Gesamtwiderstandes kann nun die Reihenschaltung zwischen $R_{2,3}$ und $R_1$ berechnet werden. Schließlich kommt ein Widerstand für die gesamte Schaltung
heraus, über dem die angelegte Spannung abfällt.\\
Eine weitere wichtige Formel zur Berechnung von Gleichstromkreisen, ist die Knotenregel. Diese findet sich auch im 1. Kirchhoffschen Gesetz wieder und sagt 
aus, dass an jedem Knotenpunkt in einem Stromnetz gleichviele Ströme hinein-, als auch wieder hinausfließen. So kann an jedem Knotenpunkt, welcher nicht 
die gleichen Ströme wie ein anderer Knoten hat, eine Knotengleichung aufgestellt werden. Mit Hilfe dieser Gleichungen lässt sich anschließend ein 
Gleichungssystem lösen, was zur Lösung des Problems führen kann.
\begin{equation}
I_1+I_2+I_3=I_4+I_5
\label{eqn:1. Kirchhoffsches Gesetz}
\end{equation}
Gibt es allerdings noch eine unbekannte Variable, dann kann die Schaltung nicht alleinig mit der Knotenregel berechnet werden, sondern benötigt zusätzlich die 
Anwendung des 2. Kirchhoffschen Gesetzes, der Maschenregel. Diese Regel besagt, dass alle Spannungen in einer Masche, heißt in einem geschlossenen Stromkreis 
von Widerständen, Spannungsquellen, etc. in Summe Null ergeben. In Kombination mit der Knotenregel kann nun fast jedes einfachere Problem in einem 
Gleichstromkreis gelöst werden.
\begin{equation}
U_1+U_2+U_3-U_4-U_5=0
\label{eqn:2. Kirchhoffsches Gesetz}
\end{equation}
Dieses Verhalten von Widerständen in Bezug auf Strom und Spannung kann durch einfache Versuche nachgewiesen werden. Einer dieser Versuche wäre \zB, dass 
man einen einfachen Stromkreis aufbaut, der einen Widerstand und einen Verbraucher \zB eine Glühbirne beinhaltet. Variiert man nun mit der Größe des 
Widerstandes, kann man bei gleichbleibender Spannung feststellen, dass die Glühbirne dunkler wird, je größer der Widerstand wird.\\ 
Ein weiterer Versuch kann durchgeführt werden, indem man zwei Glühbirnen beim ersten Durchgang in Reihe schaltet und beim zweiten Durchgang 
parallelschaltet. Man wird beobachten, dass die Glühbirnen bei der Parallelschaltung heller leuchten, als bei der Reihenschaltung. Dies liegt daran, 
dass in der Reihenschaltung Spannung über der ersten Glühbirne abfällt, da diese einen Widerstand im Stromnetz darstellt. Somit liegt an der zweiten 
Glühbirne eine geringere Spannung an und Folge dessen leuchtet diese weniger. Bei einer Parallelschaltung ist dies nicht der Fall, da dort an jeder 
Glühbirne gleichviel Spannung anliegt. Es sinkt lediglich der Strom an jeder Glühbirne.\\\\ %cite 5
Das nächste Thema im Bereich Elektrotechnik ist der Wechselstrom bzw. Drehstrom. Diese Art des Stroms hat einen großen Anwendungsbereich im deutschen 
Stromnetz, aber wird auch in jedem Haushalt oder Firma verwendet. Um nun mit diesem sicher umgehen zu können, sei es bei Reparaturen im Stromnetz oder
 bei der alltäglichen Verwendung von Haushaltgeräten, muss es Fachkräfte geben, die sich um die ordnungsgemäße Installation und den Bau kümmern. 
 Dazu werden Sie immer zu den aktuellen Sicherheitsstandards informiert und werden gegebenenfalls nachgeschult, \zB im Bereich Arbeiten unter Spannung (AuS) 
 für Stromnetz Monteure.
\paragraph{Elektroinstallationsschaltungen und Elektrogeräte}\mbox{}\\
Unter dem Begriff Elektroinstallationsschaltungen werden die meisten Menschen nichts verstehen, da Sie mit der Materie wenig zu tun haben. Allerdings ist dieser 
Bereich in jedem normalen Haushalt zu finden, da Sie meist Steckdosen und Lichtschalter für Deckenlampen in Ihrem Haus haben. Diese Installation der Kabel, 
Lampen, Steckdosen und Lichtschalter werden in der Regel von einem ausgebildeten Elektriker durchgeführt. Hierbei wird meist eine Leitung mit drei oder 
fünf Adern des Typs NYM-J vom Sicherungskasten aus verlegt und durch Fehlerstromschutzschalter (FI) / Fehlerstrom-Schutzeinrichtungen (RCD) abgesichert. 
Diese haben die wichtige Aufgabe den Stromfluss zu unterbrechen, sobald ein Strom, \zB bei einem defekten Haushaltsgerät aus Metall, über den Schutzleiter (PE) 
abfließt. Würde es keinen PE-Leiter oder FI geben, dann ist es sehr wahrscheinlich, dass der Strom über den Menschen fließt und dieser einen Stromschlag 
erleidet und gesundheitliche Probleme bekommt. Deshalb ist es auch von hoher Relevanz, dass Elektroinstallationsschaltungen von einem ausgebildeten Elektriker 
installiert werden, um zu gewährleisten, dass alle Kabel ordnungsgemäß angeschlossen wurden. Zusätzlich prüft dieser mit geeichten Messgeräten, ob die 
FIs und/oder RCDs an denen die Steckdosen angeschlossen sind bei den vorgeschriebenen Werten auslösen.\\
Beim Bau von Elektrogeräten oder Verlängerungsleitungen ist es ebenfalls wichtig diese vor Verkauf und Inbetriebnahme zu prüfen, da nie zu einhundert 
Prozent sicher ist, ob alle Bauteile nach der Produktion intakt sind. Hier muss zuerst eine Sichtkontrolle auf Beschädigung nach Protokoll durchgeführt 
werden, da Haushaltsgeräte keine FIs oder RCDs verbaut haben. Anschließend werden die Geräte an ihren Steckern mit Messgeräten gemessen, um festzustellen 
ob sie alle Grenzwerte einhalten. Hierbei wird vor allem auf den Isolationswiderstand, die Fehlerschleifenimpedanz und die Auslösezeit geachtet.\\
Ein weiterer Anwendungspunkt von RCD Messungen betrifft die Monteure des Stromnetzes, da diese an Standorten wo jährliche Volksfeste oder Märkte stattfinden 
Kabelverteilerschränke (KVS) haben, in denen Sie Steckdosen für Starkstromanschlüsse des Typs CEE haben. Diese Steckdosen sind durch RCDs abgesichert und 
müssen vor Benutzung auf Funktion und Grenzwerteinhaltung geprüft werden. Andernfalls darf dieser nicht verwendet werden, da er ein Sicherheitsrisiko 
darstellt. Hierzu wird jede CEE-Steckdose mit einem Messadapter verbunden und anschließend mit einem Installationstester gemessen. Wichtig sind bei solchen 
Steckdosen die Messung der Fehlerschleifenimpedanz $I_{\Delta \text{N}}$, welche Aussage darüber trifft, wie hoch der tatsächliche Auslösestrom im Vergleich zum 
angegebenen Auslösestrom ist. Dies ist von Wichtigkeit, um Personen zu schützen, die mit eventuell defekten Geräten arbeiten und in Gefahr laufen einen 
Stromschlag zu bekommen. Anschließend wird noch die Auslösezeit gemessen, um zu prüfen, ob der RCD schnellgenug bei einem möglichen Fehlerstrom auslöst.
Dieser Wert liegt normalerweise im zweistelligen Millisekunden Bereich und kann je nach RCD und alter unterschiedlich sein. Hierbei ist nur wichtig, dass 
der Grenzwert von 200 ms nicht überschritten wird, da sonst akute Gefahr herrscht bei Nutzung defekter Geräte. %cite 6
\begin{description}
\item[KVS] Ein KVS ist ein Schrank, der wie eine Weiche im Stromnetz funktioniert. In diesem kommen mehrere Kabel des Stromkreises an und können beliebig 
miteinander verschalten werden. Zudem können freie Leisten genutzt werden, um \zB an Volksfesten Fahrgeschäft anzuschließen und mit passenden Sicherungen 
an das Stromnetz anzubinden.
\end{description}

\section{Reflexion und Bewertung}

Die Problemstellungen aus den Bereichen Metallbearbeitung und Elektrotechnik führten zu zahlreichen Methoden der Lösung einfacher Probleme. Dazu wurden 
Verfahren angewandt, welche in der Praxis alltäglich und teils vollautomatisiert verwendet werden. Dazu zählen vor allem die Verfahren aus dem Bereich 
der Metallbearbeitung, wie \zB dem Feilen, Fräsen und Drehen, welche für vielfältige Problemstellungen angewandt werden kann. Zudem kann dieses Verfahren
 mit mathematischen Theoremen verknüpft werden, um \zB durch numerische Verfahren, wie in der CNC Technik, Werkstücke präzise zu fertigen. Somit ist der 
 Anspruch an hoch qualitativ und quantitativer Arbeit erfüllt. Ein großes Problem dabei ist nur, dass für solch komplexe Maschinen meist Personal mit 
 hoher Fach Expertise und langjähriger Erfahrung benötigt wird. Um dieses Problem ein wenig einzudämmen gibt es auch noch die genannte Methode des 
 kommerziellen Fräsens, in der keine Kenntnisse zu mathematischen Theoremen vorausgesetzt werden. Die Qualität des Endproduktes liegt hierbei in der 
 Hand des Mitarbeitenden und seiner Konzentration, da es bei kleinen Unaufmerksamkeiten schnell zu Fehlern oder Abweichungen im Werkstück kommen kann. 
 Dies führt wiederum zu einer Steigerung der nicht brauchbaren Werkstücke, auch genannt als Ausschuss.\\
Auch in der Elektrotechnik ist es von Wichtigkeit bestimmte Verfahren zur Bearbeitung von Aufgaben kennenzulernen, da dies ebenfalls das Verständnis 
für bestimmte Lösungswege fördert. Dieses Verständnis kann anschließend genutzt werden neue Probleme zu lösen und einen zielgerichteten und nachhaltigen 
Lösungsweg anzustreben. Dies konnte vor allem gezeigt werden anhand des Beispiels der RCD Messung, da diese nicht nur in der Theorie und in der Produktion 
genutzt wird, sondern auch wichtig für die Arbeit im alltäglichen Betrieb ist. Diese Maßnahmen werden fast täglich von Elektrikern verwendet und 
garantieren die Sicherheit eines jenes Bürgers. Zudem fördert das Wissen über die Vorgehensweise einer Hausanschlussschaltung die Ingenieursfähigkeiten 
und führt zur Entwicklung eines praktischen Verständnisses. Dies kann im beruflichen Alltag helfen theoretische Aufgaben qualitativer in praktische 
Umzusetzen und ein zielgerichteteres Vorgehen bei spontan auftretenden Problemen zu fördern.


\clearpage
\chapter{Montage von Baustromanschlüssen}
\label{cha:Baustromanschlüsse}

%... Theoretische Grundlagen (vielleicht auch zitiert aus Standardwerken, wie z.B. aus \autocite{Tipler.2019}), Rechercheergebnisse, Stand der Technik \index{Stand der Technik} (ggf. zitiert aus Hochschulschriften, welche %Online verfügbar sind, wie z.B.~\autocite{Ziegler.2017}), etc.

\section{Aufgabenstellung}

Hier Text ...

\section{Praktischer Lösungsweg}

Hier Text ...

\section{Reflexion und Bewertung}

Die Kontrolle und Pflege des Nieder- und Mittelspannungsnetzes der TWS Netz GmbH hat eine große Bedeutung für die Versorgungssicherheit im gesamten Versorgungsgebiet. Wie vorherig erläutert, benötigt es eine strukturierte und regelmäßige Begehung der kritischen Punkte im Netz. Dazu zählen vor allem die Freileitungsmasten aus Holz, wie auch die Ust., da Sie die Knotenpunkte im gesamten Netz darstellen. Durch die Versorgung der Ust. mit Mittelspannung, ist es überhaupt möglich ein Niederspannungsnetz zu betreiben und es wäre fatal, wenn dieses durch Banalitäten ausfällt. Um dies zu vermeiden werden regelmäßige Kontrollen und Maßnahmen ergriffen, um \zB Feuchtigkeit, Tiere und Dreck fernzuhalten. Zudem ist es wichtig, dass Netz fortschreitend zu erneuern, um Schwachstellen früh genug zu erkennen und zu beseitigen. Aber nicht nur die Betreuung der Mittelspannung ist von Relevanz, sondern auch die Pflege des Niederspannungsnetzes, da hier die Verbraucher direkt angeschlossen sind und am schnellsten von Ausfällen mitbekommen. Dies wird durch die gezielte Pflege, Kontrolle und Instandhaltung erreicht, um den Kunden immer eine funktionierende Stromversorgung zu gewährleisten. Diese ist wichtig, um ein positives Image und eine zukunftsfähige Wirtschaftlichkeit zu erreichen. Andernfalls kommt es zu einem Kundenrückgang, welcher die Zukunft des Unternehmens gefährdet. Des Weiteren sind Kontrollen bei \ce{SF_6} Anlagen enorm wichtig, da es sich um umweltschädliche Substanzen handelt und diese Auflagenkonform betrieben werden müssen, um die daraus folgenden Umweltbelastungen zu minimieren. 

\clearpage

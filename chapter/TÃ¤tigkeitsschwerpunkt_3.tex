\chapter{Kabelmuffen im Nieder- und Mittelspannungsnetz}
\label{cha:Kabelmuffen}

\section{Aufgabenstellung}

Zu den Haupttätigkeitsbereichen des Betrieb Stromnetzes gehören die Kabelverbindungen, Kabelabzweige und Kabelenden, auch Muffen genannt. Diese bringen 
die Aufgaben der Installation mit sich und können je nach Anwendungsbereich verschiedene Arten der Installation aufweisen. Dazu zählen \zB die 
Verbindungsmuffen, die Abzweigmuffen, aber auch spezielle Übergangsmuffen oder Kabelenden im Bereich der Nieder- und Mittelspannung. Zudem unterscheiden 
sich die Kabelverbindungen im Niederspannungsnetz, mit denen im Mittelspannungsnetz, da dort viel höhere Anforderungen an die Verbindungen gestellt 
werden und eine höhere Sicherheit vonnöten ist. Um solch eine Qualität zu gewährleisten muss vorausgesetzt werden, dass jeder Mitarbeiter über die 
Vorgehensweise und den Umgang mit Material, sowie mit Gefahrstoffen informiert ist und dies bei seinem Problem anwenden kann. Diese Problemstellungen 
können sich unterscheiden von einem einfachen verbinden zweier Kabel, über den Übergang von einem dünnen auf ein dickeres Kabel, wie auch ein Abzweig 
von einem auf zwei neue Kabel. Bei Mittelspannungskabeln fällt der Abzweig weg, da dies technisch nicht möglich ist und somit in einem Schaltwerk oder 
in einer Ust. durch Lasttrennschalter erfolgt. Zudem fallen auch Kabelenden in den Bereich der Kabelmuffen. Hierbei wird zusätzlich unterschieden 
zwischen spannungsfesten und spannungsfreien Kabelenden. 

\section{Praktische Lösung}

Die erste Art der Kabelmuffen, ist die Verbindungsmuffe. Diese dient zur unterbrechungsfreien Verbindung zweier Kabel und findet meist ihren Einsatzbereich 
in der Verlängerung oder Reparatur vorhandener Kabel. Zudem kann diese Art der Muffe flexibel eingesetzt werden und bietet zwei verschiedene Methoden zur 
Montage. Eine dieser Methoden ist die Warmschrumpftechnik, in der die zusammengefügte Stelle mit Hilfe von Schrumpfschläuchen isoliert wird. Der Begriff 
warmschrumpfen kommt vom Schrumpfen der Schläuche durch Hitze. Dieses sogenannte Schrumpfen beschreibt den Prozess, in dem sich der Kunststoffschlauch 
aufgrund seiner chemischen Eigenschaften als Thermoplaste zusammenzieht und nach abkühlen seine Form beibehält. Diese Eigenschaft der Umformbarkeit bei 
Wärmezufuhr beschreibt die thermoplastischen Kunststoffe. Um nun die beiden Kabel zu verbinden, werden sogenannte Schraubverbinder eingesetzt. Diese können 
auf ein abisoliertes Kabelende geschraubt werden und stellen eine Verbindung zwischen den Kabeln her.
%\begin{figure}[hbt]
%\centering
%\includegraphics[width=0.8\linewidth]{images/Schraubverbinder NS}
%\caption[Schraubverbinder NS]{Schraubverbinder Niederspannung}
%\label{fig:Schraubverbinder NS}
%\end{figure}
Der auf dem Bild zu sehende Schraubverbinder, wird ausschließlich im Bereich der Niederspannung eingesetzt, da er nur einsetzbar bis zu einem maximalen 
Aderquerschnitt von 150 $\text{mm}^2$ ist. Dies ist völlig ausreichend, da ein Aderquerschnitt von 150 $\text{mm}^2$ der größte im Niederspannungsnetz der 
TWS Netz GmbH darstellt und eine solche Muffe nur bis 1 kV betrieben werden darf. Diese Schraubverbinder werden auf alle vier Adern eines Erdkabels des 
Typs NAYY geschraubt und anschließend mit separaten Schrumpfschläuchen isoliert, um einen Kurzschluss zwischen den Leitern zu verhindern. Dieser Erdkabeltyp 
besteht aus vier Aluminiumleitern, welche einzeln isoliert sind und durch eine zusätzliche Füllung zwischen Außenmantel und Aderisolierung vor Verdrehung 
geschützt werden.%cite 7.0
Um den Außenmantel zu ersetzen, wird bei einer Verbindungsmuffe ein großer Schrumpfschlauch über beide Kabel abgeschrumpft, um das gesamte Muffenpaket zu 
schützen, wenn es in der Erde liegt. Das Muffenpaket definiert sich aus dem Bündel der vier Schraubverbinder einer Verbindungsmuffe, den abgemantelten 
Adern der Kabel und dem darüber abgeschrumpften Mantelschlauch. %cite 7 
Die Verwendung von Aluminiumkabeln ist geläufiger, als die von Kupferkabeln, da es wesentlich günstiger und auch deutlich leichter im Gewicht ist. Die Dichte 
von Aluminium liegt bei 2,71 $\frac{\text{g}}{\text{dm}^3}$ und Kupfer hat fast das 3,5 fache der Dichte, mit einem Wert von 
8,92 - 8,96 $\frac{\text{g}}{\text{dm}^3}$. %cite 8
Rechnet man dies um auf ein 100 Meter langes Stück des Kabeltyps NAYY, dann ergibt sich eine Differenz des Gewichtes von ca. 350 Gramm. Dazu kommt noch das 
Gewicht der Isolation, welches allerdings ähnlich bei beiden Typen ist. Preislich ist das Kupferkabel um das 5 fache teurer und ist somit nicht wirtschaftlich 
genug im Bezug auf den einzigen Nachteil, dass der Leitungswiderstand geringfügig kleiner ist beim Aluminiumkabel.%cite 9 
Dieser Leitungswiderstand kann berechnet werden, mit Hilfe des spezifischen Widerstandes, dem Querschnitt und der Länge des Kabels. Der spezifische Widerstand 
ist eine konstante Größe für unterschiedliche Materialien und kann in folgender Tabelle abgelesen werden. 
\begin{table}[hbt]	
	\centering
	\renewcommand{\arraystretch}{1.5}
	\captionabove[Materialkonstanten]{Materialkonstanten \autocite{Tipler.2019}}
	\label{tab:Materialkonstanten }
	\begin{tabular}{|c|c|c|c|c|}
        \hline
		\textbf{Material} & \textbf{Symbol} & \textbf{spez. Widerstand} & \textbf{spez. Leitwert} & \textbf{Temperaturkoeffizient}\\
        \textbf{} & \textbf{} & \textbf{in} $\mathbf{\frac{\Omega \cdot \textbf{mm}^2}{\textbf{m}}}$ & \textbf{in} $\mathbf{\frac{\textbf{m}}{\Omega \cdot \textbf{mm}^2}}$ & \textbf{in} $\mathbf{\frac{1}{^\circ \textbf{C}}}$ \textbf{oder} $\mathbf{\frac{1}{\textbf{K}}}$\\
		\hline 
		Aluminium   & \ce{Al}               &   0,028 &   36    & 0,004     \\
		\hline 
        Silber      & \ce{Ag}               &   0,016 &   63    & 0,004     \\
        \hline
        Kupfer      & \ce{Cu}               &   0,018 &   56    & 0,004     \\
        \hline
        Gold        & \ce{Au}               &   0,023 &   44    & 0,004     \\
        \hline
        Platin      & \ce{Pt}               &   0,11  &   9     & 0,002     \\
        \hline
        Eisen       & \ce{Fe}               &   0,125 &   8     & 0,005     \\
        \hline
        Manganin    & \ce{Cu, Fe, Mn, Ni}   &   0,4   &   2,5   & 0,00001   \\
        \hline
        Chromnickel & \ce{Cr, Ni, Fe}       &   1     &   1     & 0,00005   \\
        \hline
	\end{tabular} 
\end{table}
Anschließend kann mit Hilfe der folgenden Formel ein Leitungswiderstand für unterschiedliche Materialien, Querschnitte und Längen berechnet werden.
\begin{equation}
R_a=\frac{\rho \cdot l}{A}=\frac{l}{\kappa \cdot A}
\label{eqn:Leitungswiderstand}
\end{equation}
Rechnet man nun den Unterschied zwischen einem Aluminium- und Kupferkabel aus, kommt man zu einem so geringfügigen Ergebnis, dass es nicht rentabel ist, 
Kupferkabel weiter zu verwenden.


\section{Reflexion und Bewertung}

Hier Text ...

\clearpage

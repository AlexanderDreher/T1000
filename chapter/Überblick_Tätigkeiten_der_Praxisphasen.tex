\chapter{Einführung in die Tätigkeiten und Strukturen der TWS Netz GmbH}

Die Technischen Werke Schussental GmbH \& Co. KG (TWS) sind ein ökologisch orientiertes und regional verwurzeltes Unternehmen, das sich seit 2001 um die 
Versorgung der Bürger in der Region mit Strom, Gas, Wasser und Wärme kümmert. Entstanden ist die TWS aus einer Fusion der Stadtwerke Ravensburg und 
Weingarten. Im Strombereich bietet die TWS seit 2008 nur Produkte aus regenerativen und erneuerbaren Energiequellen an. So trägt das Unternehmen zum 
Klimaschutz und zur Vermeidung von CO2 bei.
\\\\
Im ersten Teil der Praxisphase wurden grundlegende Fertigkeiten zur Metallbearbeitung erlernt. Diese sind hinreichend von der Bearbeitung mit der Hand, 
als auch mit der Maschine. Dabei war das Ziel ein Werkstück zu erstellen, durch Tätigkeiten wie Feilen, Sägen, Biegen, Bohren, Drehen, Fräsen oder Schweißen. 
Anschließend wurden grundlegende Fertigkeiten im Bereich Elektrotechnik erlernt. Diese wurden geschult durch das selbständige errechnen, aufbauen und messen 
von Gleichstromkreisen, als auch das installieren von Elektroinstallations- und Schützschaltungen im Bereich der Wechselstromlehre.
\\\\
In der zweiten Praxisphase wurden Betriebsaufgaben im Bereich der Stromversorgung der TWS Netz GmbH kennen gelernt. Hierbei wurden die wichtigsten technischen
und organisatorischen Prozesse der Stromversorgung dargelegt und deren Funktionen erläutert. Dies sind zum Beispiel Änderungen am Schaltfeld, 
Mittelspannungsschaltungen oder das Schneiden von inaktiven Mittelspannungskabeln. Außerdem gehören auch Tätigkeiten wie das Freischneiden von Freileitungen 
oder das entfernen von Verunreinigungen oder Wasser aus Umspannstationen dazu. Weitere Tätigkeiten des Betrieb Stromnetzes sind die Prüfung von neuen 
Mittelspannungskabeln, der Anschluss und die Inbetriebnahme von neuen Kabelverteilerschränken und Umspannstationen, das Installieren von Verbindungsmuffen 
oder Abzweigmuffen und das kontrollierenvon Freileitungsmasten. Bei diesen Tätigkeiten kann es zu ausfällen im Stromnetz kommen, welche immer durch 
geeignete Zuschaltung von Umleitungen oder dem Einsatz von Stromaggregate größtenteils vermieden werden soll. 

\clearpage

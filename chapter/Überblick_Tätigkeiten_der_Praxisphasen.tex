\chapter{Einführung in die Tätigkeiten und Strukturen der TWS Netz GmbH}

Die Technischen Werke Schussental GmbH \& Co. KG (TWS) sind ein ökologisch orientiertes und regional verwurzeltes Unternehmen, das sich seit 2001 um die 
Versorgung der Bürger in der Region mit Strom, Gas, Wasser und Wärme kümmert. Entstanden ist die TWS aus einer Fusion der Stadtwerke Ravensburg und 
Weingarten. Im Strombereich bietet die TWS seit 2008 nur Produkte aus regenerativen und erneuerbaren Energiequellen an. So trägt das Unternehmen zum 
Klimaschutz und zur Vermeidung von \ce{CO_2} bei. 
\\\\
In den ersten beiden Praxisphasen wurde hierbei der Fokus auf das Kennenlernen der internen Strukturen und Prozesse der Stromversorgung gelegt, sowie die 
grundlegende Schaffung von Kenntnissen im Bereich der Metallverarbeitung und der Elektrotechnik. 
\\\\
Zu diesen Tätigkeiten zählen allgemeine Betriebsaufgaben wie die Überprüfung vorhandener Anlagen auf einen einwandfreien und betriebssicheren Zustand. 
Dabei werden Instandsetzungsaufgaben nötig die unter anderem auch durch gemeldete Schäden und Mängel dritter zustande kommen. Weitere Aufgaben sind im 
Zusammenhang mit Neubau-, Instandsetzungs- oder Fremdaufträgen verbunden. Diese umfassten Arbeiten im Bereich der Kabelverbindungen sowie die Implementierung 
effizienter Netzwerkmanagementstrukturen in der TWS Netz GmbH. 
\\\\
Des Weiteren wurden Kenntnisse im Bereich Metallverarbeitung und Elektrotechnik erworben, um ein breiteres Verständnis für verschiedene Problemlösungsmethoden 
zu entwickeln. Dies schloss das Kennenlernen verschiedener Metallbearbeitungstechniken ein, die in praktischen Aufgabenstellungen angewendet wurden. Im 
Bereich der Elektrotechnik wurden Kenntnisse in Gleich- und Wechselstromlehre durch die Bearbeitung von Aufgaben unter Anleitung erworben und anschließend 
in praktischen Experimenten umgesetzt und bewiesen.

\clearpage

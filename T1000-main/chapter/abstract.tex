\chapter*{Überblick Tätigkeiten der Praxisphasen} %*-Variante sorgt dafür, das Abstract nicht im Inhaltsverzeichnis auftaucht

Die Spannung entsteht durch Thermodiffusionsströme in einem Material. Die Betrachtung nur eines Materials mit Temperaturgradienten liefert also eine hinreichende Erklärung. Die entstehende Spannung (Integral des elektrischen Feldes) ist die Seebeck-Spannung. Für Messzwecke braucht man zwei verschiedene Metalle. Am heißen Ende des Leiters gibt es mehr Elektronen mit hoher Energie und weniger Elektronen mit geringer Energie (unterhalb des chemischen Potenzials). Durch Diffusion bewegen sich entsprechend energiereiche Elektronen zum kalten Ende und Elektronen mit wenig Energie in die entgegengesetzte Richtung. Dies beschreibt die Wärmeleitung durch Elektronen. Ein eventuelles Ungleichgewicht der Ströme wird durch ein elektrisches Feld ausgeglichen, da im offenen Stromkreis kein Strom fließen kann.

Die Seebeck-Spannung wird durch die Abhängigkeit der Beweglichkeit und Anzahl (Zustandsdichte) der Elektronen von der Energie bestimmt. Die Abhängigkeit der Beweglichkeit von der Energie hängt empfindlich von der Art der Streuung der Elektronen ab. Entsprechend können auch relativ kleine Verunreinigungen die Thermospannung recht stark beeinflussen. Die treibende Kraft für die Diffusion ist näherungsweise proportional zur Temperaturdifferenz. Als grober Trend für Metalle nimmt der Seebeck-Koeffizient mit steigender Temperatur zu. Die örtliche Verteilung des Temperaturgefälles längs der Leitung ist ohne Bedeutung.

Ein Spezialfall ist der so genannte Elektronen-Drag. Bei niedrigen Temperaturen von etwa 1/5 der Debye-Temperatur werden die Phononen vor allem durch Stöße mit Elektronen gebremst. Die Phononen ziehen dabei die Elektronen mit in Richtung niedriger Temperaturen. Dadurch können in diesem Temperaturbereich die thermoelektrischen Effekte etwas größer werden, als man es sonst erwartet. Bei höheren Temperaturen gewinnen Umklappprozesse für die Streuung der Phononen an Bedeutung und der Effekt wird kleiner.
\cleardoublepage
